\documentclass[grado3]{LEMA-Tikz-IM}
\begin{document}
\begin{tikzpicture}
% \usetikzlibrary{positioning} % for relative note placement (right = of)

\tikzset{%
  node distance=0cm   % 0 distance between nodes so that edges coincide
}

% Style of "tarjeta" node
\tikzstyle{tarjeta}=[%
    , draw                 % draw rectangle edges     
    , text width = 2.7in     % min width     
    , minimum height = 2.6in % min height
    , align = center       % alignment
    , outer sep = 0cm      % 0 "personal space" for nodes
    , inner sep = 2ex      % 0 "personal space" for nodes
    , execute at begin node={\centering\let\raggedright\relax} % Prevent hyphenation and center
    % , text height=0.5cm    % vertical placement of text inside node
    % , text depth=0.0cm,    % vertical placement of text inside node (modes opposite dir)
]


\def\textoTarejetaA{El grillo topo tiene varias patas. Algunas de ellas son especiales para cavar. En un grupo de 5 grillos topo se contaron diez patas especiales para cavar. ¿Cuántas patas especiales tiene cada grillo?}
\def\textoTarejetaB{Un escarabajo tiene un par de antenas para detectar el calor, tocar, oler y otras cosas más. Si hay 8 antenas, ¿cuántos escarabajos hay?}
\def\textoTarejetaC{En un grupo de abejas alguien contó catorce antenas. Si cada abeja tiene 2 antenas, ¿cuántas abejas hay?}
\def\textoTarejetaD{En un grupo de libélulas hay 12 alas. Si cada libélula tiene 4 alas, ¿cuántas libélulas hay?}
\def\textoTarejetaE{Alguien contó treinta patas en un grupo de 5 hormigas. Si todas las hormigas tienen el mismo número de patas, ¿cuántas patas tiene cada hormiga?}
\def\textoTarejetaF{En total hay 50 manchas en 5 mariposas. Si todas las mariposas tienen el mismo número de manchas, ¿cuántas manchas tiene cada mariposa?}


\node[tarjeta] (tarjetaA) {\textoTarejetaA};
\node[below right] at (tarjetaA.north west) {A};

\node[tarjeta, below = of tarjetaA] (tarjetaB) {\textoTarejetaB};
\node[below right] at (tarjetaB.north west) {B};

\node[tarjeta, below = of tarjetaB] (tarjetaC) {\textoTarejetaC};
\node[below right] at (tarjetaC.north west) {C};


\node[tarjeta, right = of tarjetaA] (tarjetaD) {\textoTarejetaD};
\node[below right] at (tarjetaD.north west) {D};

\node[tarjeta, below = of tarjetaD] (tarjetaE) {\textoTarejetaE};
\node[below right] at (tarjetaE.north west) {E};

\node[tarjeta, below = of tarjetaE] (tarjetaF) {\textoTarejetaF};
\node[below right] at (tarjetaF.north west) {F};


\foreach \tarj in {A,...,F}{
  \node[below left, font=\scriptsize] at (tarjeta\tarj.north east){Todo sobre bichos};
}


\end{tikzpicture}
\end{document}